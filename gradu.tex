% !TeX encoding = latin1
%
% [ Tiedostossa k�ytetty merkist� on ISO 8859-1 eli Latin 1. Yll�oleva rivi ]
% [ tarvitaan, jos k�ytt�� MiKTeX-paketin mukana tulevaa TeXworks-editoria. ]
%
% Laatinut Timo M�nnikk�
%
% Jos kirjoitat pro gradu -tutkielmaa, tee mallipohjaan seuraavat muutokset:
%  - Poista dokumenttiluokasta optio shortthesis .
%  - Poista makro \tyyppi .
%  - Lis�� suuntautumisvaihtoehto makrolla \linja .
%  - Kirjoita ylimm�n tason otsikot makrolla \chapter, toisen tason otsikot
%    makrolla \section ja mahdolliset kolmannen tason otsikot makrolla
%    \subsection .

\documentclass[finnish,nonumbib,nocopyright]{gradu2}

\usepackage{palatino} % valitaan oletusfonttia hieman tyylikk��mpi fontti

\usepackage[dvips]{graphicx} % tarvitaan vain, jos halutaan mukaan kuvia

\title{BDD ohjelmistokehityksen tukena mobiiliohjelmistojen kehityksess�}
\translatedtitle{Using BDD to support software development on mobile platforms}

\setauthor{Samppa}{Hynninen}
\yhteystiedot{samppa.hynninen@jyu.fi}

\setdate{12}{02}{2013}

\tiivistelma{T�h�n tulee tutkielman tiivistelm�.}
\abstract{Here comes the abstract of the thesis.}

\avainsanat{bdd, behariour-driven development, mobiilialustat}    % korvaa n�m� oikeilla
\keywords{bdd, behariour-driven development, mobile platforms} % avainsanoilla

\linja{Ohjelmisto- ja tietoliikennetekniikka}

\begin{document}

\mainmatter

\chapter{Johdanto}

Johdanto t�h�n

\chapter{Ketter� ohjelmistokehitys}

\section{Taustaa ja ketterien menetelmien kehitys}

\section{TDD}

\section{ATDD ja BDD}

\chapter{BDD:n tarjoamat liiketoiminnalliset edut}

\section{Asiakkaan k�ytt�j�tarinoista hyv�ksymistesteiksi}

\section{BDD:n hy�dynt�minen mobiilikehityksess� offshore-tiimeill�}

\chapter{BDD mobiilialustoilla}

\section{Natiivisovellukset}

\section{HTML5-sovellukset}

\chapter{Crossplatform-testaaminen eri mobiilialustoilla}

\section{iOS BDD-frameworkit}

\section{Android BDD-frameworkit}

\section{Windows phone BDD-frameworkit}

\section{Mahdollisuudet testata kaikki alustat yhdell� testisetill�}

\chapter{Yhteenveto}

Yhteenveto t�h�n

\begin{thebibliography}{9} % numero aaltosuluissa kertoo, miten paljon
                           % varataan tilaa viitenumerolle

\bibitem{agilemanifesto}
\textit{Agile Manifesto}, 2001, saatavilla WWW-muodossa
<URL: \texttt{http://agilemanifesto.org/}>, viitattu 12.02.2013.

\bibitem{cucumberbook}
Matt Wayne ja Aslak Helles�y, \textit{The Cucumber Book, Behaviour-Driven Development for Testers and Developers},
Pragmatic Programmers LLC, 2012.

\bibitem{bddcharacteristics}
Carlos Sol�s ja Xiaofeng Wang, \textit{A Study of the Characteristics of Behaviour Driven Development},
37th EUROMICRO Conference on Software Engineering and Advanced Applications, 2011.

\bibitem{bddintro}
Dan North, \textit{Introducing BDD}, 2006, saatavilla WWW-muodossa
<URL: \texttt{http://dannorth.net/introducing-bdd/}>, viitattu 27.01.2013.

\bibitem{specification}
Gojko Adzic, \textit{Specification by Example},
Manning Publications Co, 2011.

\bibitem{state}
Susan Hammond ja David Umphress, \textit{Test driven development: the state of the practice},
ACM-SE '12 Proceedings of the 50th Annual Southeast Regional Conference, s. 158-163, 2012.

\bibitem{bddintro2}
Chris Rimmer, \textit{Introduction, Behaviour-Driven Development}, 2010, saatavilla WWW-muodossa
<URL: \texttt{http://behaviour-driven.org/Introduction}>, viitattu 27.01.2013

\bibitem{ibm}
IBM developerWorks: http://www.ibm.com/developerworks/java/library/j-cq09187/index.html

\bibitem{tech}
RSpec, Cucumber, JBehave, Robotium, Selenium jne.

\end{thebibliography}

\end{document}
